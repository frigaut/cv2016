%!TEX TS-program = xelatex
\documentclass[]{friggeri-cv}
% \documentclass[print]{friggeri-cv}
\addbibresource{citations.bib}

% \usepackage[hyperref]{color}
% \usepackage[colorlinks]{hyperref}
% \AtBeginDocument

\begin{document}

\header{Fran\c{c}ois }{Rigaut}
       {Curriculum Vitae}

% In the aside, each new line forces a line break
\begin{aside}
  \section{About}
    54 Eucumbene drive
    Duffy ACT-2611
    Australia
    M: +61 (0)4 9968 0000
    W: +61 (0)2 61 250 210
    \href{mailto:francois.rigaut@anu.edu.au}{francois.rigaut@anu.edu.au}
    \href{mailto:frigaut@gmail.com}{frigaut@gmail.com}
    \href{http://www.linkedin.com/profile/view?id=147884997}{linkedin profile}
  \section{Languages}
    Fluent in French/English/Spanish
    Notions of Italian/German
  \section{Acronyms}
    {\small AO = Adaptive Optics}
    {\small MCAO = Multi-Conjugate AO}
    {\small GeMS = Gemini MCAO system}
    {\small LGS = Laser Guide Star}
    {\small GMT = Giant Magellan Telescope}
    % {\small ANU = the Australian National University}
  \section{Programming}
    {\color{red} $\varheartsuit$} Yorick
    Python, C, IDL
    CSS3 \& HTML5
\end{aside}

% \section{interests}
% complex networks, social networks, community detection, community structure,
% overlapping communities, information diffusion, viral marketing, social
% inference, recommendation, data mining

\section{Employment and education}

\begin{entrylist}
  \entry
    {since 2012}
    {Associate Professor / AO Group Leader}
    {Mount Stromlo Observatory, RSAA, ANU}
    {At the Research School of Astronomy and Astrophysics, Australian National University, I am heading the Adaptive Optics group, comprising (AO) scientists, postdocs and students. When I joined the RSAA at the ANU in 2012, work was proceeding on the Giant Magellan Telescope Laser Tomography System and the Adaptive Optics Demonstrator (in collaboration with EOS). Since then, bridging a gap in the GMT funding, we have successfully diversified our activities: We won an ARC grant for the design and construction of \textbf{NGS2}, an innovative focal plane upgrade for the GeMS Natural Guide Star Wavefront sensor; We produced a satellite imaging AO system for the Korean Astronomy Space Science Institute (\textbf{KASI}) in association with EOS; We built up comprehensive \textbf{AO} laboratory \textbf{facilities}. Most importantly, ANU is a core partner in the \textbf{Space Environment Management Cooperative Research Centre} (\href{http://www.serc.org.au}{http://www.serc.org.au}, federal funding of \$20M over 5 years). This CRC is investigating solutions to the space debris problem, which threatens the viability of the entire low earth orbit environment. Thanks to the CRC funding and internal RSAA commitments, our group is growing; we just hired two postdocs and have signed up to train six PhD students in the next four years.}
    %Because of the wide implications of the space debris issue, its challenges, and the education aspect, I very much enjoy my current work at the ANU.}
    % Our main activities are centred on the Giant Magellan Telescope (GMT) Laser Tomography system, space debris tracking and de-orbiting activities (within the Space Environment Management Cooperative Research Centre http://www.serc.org.au), commercial activities in association with EOS (Electro Optic Systems), as well as investigating avenues for innovative applications of Adaptive Optics.
  \entry
    {1999-2012}
    {Lead Adaptive Optics Scientist}
    {Gemini Observatory, Hawaii/Chile}
    {The Gemini Observatory is a multinational organisation managing two 8-m telescopes, located in Hawaii and Chile. Over the period I worked at Gemini, the total investment in AO has been close to \$40M. I was coordinating the program on the Adaptive Optics science side, consulting with our community, setting goals and requirements, facilitating the design of the instruments and commissioning them. In 2006, I moved from Hawaii to the Chile Gemini South headquarters to concentrate on the Multi-Conjugate Adaptive Optics instrument, which was successfully commissioned in 2011/2012.}
  \entry
    {1997-1999}
    {Head of the Adaptive Optics Group}
    {ESO, Munich, Germany}
    {ESO --the European Southern Observatory,-- with over 550 employees, is the largest European Organisation for Astronomy. I was heading there a 13 person group in the instrumentation division. My responsibilities, on top of the group management, included defining future AO instrumentation for the VLT (Very Large Telescope array), following instrument design and fabrication.}
  \entry
    {1992-1997}
    {Project Scientist for the Adaptive Optics Bonnette}
    {CFHT, Hawaii, USA}
    {Thanks to its ease of use, reliability and performance, PUEO, the Canada-France-Hawaii Telescope AO bonnette, has been for a long time the most successful AO instrument in the world. I heavily participated in the design phase, led the instrument commissioning, and insured that the community put it to good use.}
  \end{entrylist}
  \begin{entrylist}
  \entry
    {1989-1992}
    {PhD}
    {Observatoire de Paris-Meudon, France}
    {Under the supervision of Prof. Pierre Léna, my PhD thesis, entitled ``Applications of Adaptive Optics to Astronomy'', was one of the precursor work in this field. I worked on the first AO system dedicated to astronomy, and acted as the {\em de facto} person-in-charge for its commissioning. This allowed me to gain first hand experience on the behavior and performance of AO systems.}
  \entry
    {06/1987}
    {Master of Science}
    {Paris VII University, France}
    {"Astrophysique et Techniques Spatiales"}
  \entry
    {06/1986}
    {Bachelor of Physics (Ma\^itrise)}
    {Universit\'e Claude Bernard, Lyon, France}
    {}
\end{entrylist}

\section{Achievements and awards}


Below are career achievements I am most proud of, sorted by order of importance (to me), most important first.


\begin{entrylist}
  \entry
    {1992-today}
    {Characterisation of real-world AO systems performance}
    {}
    {{\em ``First diffraction-limited astronomical images with adaptive optics''},
    244 citations (citation counts from Google scholar, as of November 2015), {\em ``Adaptive optics on a 3.6-m telescope- Results and
    performance''}, 145 citations and {\em ``Performance of the Canada-France-Hawaii
    Telescope adaptive optics bonnette''}, 215 citations.}
  \entry
    {1999-2012}
    {GeMS}
    {}
    {GeMS, the Gemini MCAO facility instrument, is the first and only Laser Guide Star Multi-Conjugate AO system for night time astronomy. By using a combination of 5 laser guide stars, associated wavefront sensors, and 3 deformable mirrors, this system provides 10x improvement in compensated sky area compared to classical AO systems. Commissioned in 2011-2012, it is now available to the Gemini astronomical community ({\em ``Gemini multiconjugate adaptive optics system review–I and II''}). Science papers published from data obtained with GeMS can be found in the \href{http://www.gemini.edu/apps/publications-users/?Publication[instrument]=GeMS}{Gemini paper archive}.}
  \entry
    {2001}
    {Seminal paper on Ground Layer AO}
    {}
    {I invented the concept of Ground Layer Adaptive Optics (now used in several of the largest telescopes --e.g. ESO VLT, LBT-- and planned for the majority of extremely large telescopes -- the European ELT and the GMT, a total capital investment of over US\$2.6 Billions) and wrote the first paper analysing its performance potential ({\em ``Ground-conjugate wide field adaptive optics for the ELTs''}, 103 citations).}
  \entry
    {1998}
    {Analytic modelling of AO system performance}
    {}
    {I invented a novel way (Fourier analysis) of modelling AO performance, followed and built upon by many researchers ({\em ``An Analytical model for Shack- Hartmann- based adaptive optics systems''}, 97 citations).}
  \entry
    {1992}
    {First assessment of sky coverage in LGS mode and first analytical derivation of noise in Shack-Hartmann systems}
    {}
    {{\em ``Laser guide star in adaptive optics -- The tilt determination problem''}, 216 citations.}
  \entry
    {2015}
    {h-index of 38 (i-10 index of 96)}
    {}
    {According to
    \href{http://scholar.google.com/citations?hl=en\&user=SD\_leV4AAAAJ}{google scholar}. In 2008, I was listed by Google Scholar within the 4 main ``key authors'' in ``Adaptive Optics'', along Hardy, Roddier and Wizinowich (since then, Google discontinued this type of ranking).}
  \end{entrylist}
  \begin{entrylist}
  \entry
    {1994-today}
    {AO simulation packages (yao)}
    {}
    {I have written several extensive open-source AO Monte-Carlo simulation tools, used by a large fraction of the AO community (GMT, LBT, Keck, CFHT, UofH, Gemini, ESO, NSO, GTC, and many more).}
  \entry
    {1992-today}
    {Design and commissioning of AO systems}
    {}
    {I have been the instrument scientist or PI for many successful AO systems: COME-ON (ESO), PUEO (CFHT), Altair (Gemini) and MCAO (Gemini). These instruments represent a cumulative investment of (2015\$) US\$32.4M (involved as PI/Project Scientist) and US\$16M (involved in a managerial position at ESO).}
  \entry
    {1994-today}
    {Teaching \& mentoring}
    {}
    {I have co- directed the PhD thesis of Olivier Lai (defense in 1996), Jean- Pierre V\'eran (defense in 1997) and Ralf Flicker (defense in 2003). All obtained their PhD with honours and the first two now hold permanent positions in France and in Canada. I have also mentored a dozen of short term/summer students in the past 10 years.}
  \entry
    {1996-today}
    {Review panels}
    {}
    {Chair or member of numerous review panels, including E-ELT telescope
    readiness reviews, GMT AO, LBT ARGOS, CANARY \& many AO systems}
\end{entrylist}

% \vspace*{3mm}
{\bf Awards}

\begin{entrylist}
  \entry
    {2012}
    {AURA Technology/Innovation \href{http://www.aura-astronomy.org/news/awards.asp}{Award}}
    {}
    {Granted for the Gemini MCAO system first light}
  \entry
    {2012}
    {AURA Team \href{http://www.aura-astronomy.org/news/awards.asp}{Award}}
    {}
    {Granted to the GeMS team for the Gemini MCAO system first light}
  \entry
    {1999}
    {AURA Science \href{http://www.aura-astronomy.org/news/awards.asp}{Award}}
    {}
    {Granted for the Gemini MCAO system design concept}
\end{entrylist}


\section{Experience, skills \& hobbies}

{\bf Instrumentation}

Detailed knowledge and extensive experience in optics, control systems, complex systems, computer hardware, detectors. Broad knowledge in electronics and mechanics.

{\bf Computers and coding}

I have extensive experience in data analysis, computer simulations and associated languages (\href{http://www.maumae.net/yorick}{Yorick}, IDL, some iraf, see the \href{http://www.maumae.net/yao}{yao website} for an example of an AO simulation package I released), web design (html, php) and administration, Unix, Linux, MacOsX. I am an active proponent of open-source software and a co-admin of sourceforge.net project (yorick). I am also proficient in linux, including system administration and package management; I have maintained ubuntu, fedora, mandriva, and archlinux repositories.

{\bf Other interests}

Cooking, cycling, motorcycling, reading, listening to classical, electronica and world music, diving, hiking, skiing, helping others with computers.

I am also the team leader of my local Community Fire Unit in Canberra Duffy, an organisation of volunteers that provide help and support to fire crews.


\newpage
\section{Complete publication list}

% In the following, I listed authored or co-authored publications with more than 38 citations (according to google scholar). They are

Sorted according to the number of citations
(not reported here, see \href{http://scholar.google.com/citations?hl=en\&user=SD\_leV4AAAAJ}
{google scholar} for details). % I have over 260 publications in total.


% \printbibsection{article}{articles in peer-reviewed journal}
% \printbibsection{inproceedings}{articles in conference proceedings}

\begin{refsection}
  \nocite{*}
  \newrefcontext[sorting=none]
  \printbibliography[type=article, title={Refereed journals}, heading=subbibliography]
\end{refsection}

\begin{refsection}
  \nocite{*}
  \newrefcontext[sorting=none]
  \printbibliography[type=inproceedings, title={Conference proceedings}, heading=subbibliography]
\end{refsection}

% \begin{refsection}
%   \nocite{*}
%   \printbibliography[sorting=chronological, type=inproceedings, title={conferences proceedings}, keyword={france}, heading=subbibliography]
% \end{refsection}
% \printbibsection{misc}{other publications}
% \printbibsection{report}{research reports}

% \newpage
% \section{Referees}
%
% \textbf{Dr Matt Mountain}
%
% Matt Mountain was director of Gemini when I joined in 1999. For a number of years --after the passing of Dr Fred Gillett-- he was my immediate supervisor there. Matt left Gemini in 2005 to take the directorship of the STScI. Since March 2015, Matt has been director of the Association of Universities for Research in Astronomy (AURA).
%
% \href{mailto:mmountain@aura-astronomy.org}{mmountain@aura-astronomy.org}\\
% Phone: +1 202 204 1372
%
% \textbf{Professor Claire Max}
%
% I first met Claire Max when we were both newcomers to the field of Adaptive Optics, sometime around 1991, in Tucson. Since then, Claire and I have been regularly meeting and interacting at many adaptive optics and astronomical instrumentation conferences, as well as a few project reviews. We are currently working on a review of Adaptive Optics for Astronomy, for publication in PASP. Since 2015, Claire has been director of the University of California Observatories.
%
% \href{mailto:max@ucolick.org}{max@ucolick.org}\\
% Phone: +1 831 459 2991
%
% \textbf{Dr Doug Simons}
%
% I have known Doug Simons since we were resident astronomers at CFHT in 1992-1994. Doug hired me at Gemini in 1999, where we worked together until I came to the ANU in 2012. Doug was the Gemini director from 2006 to 2012. He is now director of the Canada--France--Hawaii Telescope.
%
% \href{mailto:simons@cfht.hawaii.edu}{simons@cfht.hawaii.edu}\\
% Phone: +1 808 885 7944
%
% \textbf{Professor Roberto Ragazzoni}
%
% I have known Roberto Ragazzoni since the early 1990s. Our paths have crossed many times since then. He is currently a full professor at the Observatory of Padova, Istituto Nazionale di Astrofisica (INAF).
% Roberto is undoubtedly one of the tall figure in the field of adaptive optics. %; arguably the most innovative and creative one too:
% He invented many concepts, such as layer-oriented multi-conjugate wavefront sensing, multiple field of view wavefront sensing for multi-conjugate adaptive optics, tip-tilt retrieval for laser guide star systems and, most importantly, the pyramid wavefront sensing now widely used and arguably the best wavefront sensing method for adaptive optics.
%
% \href{mailto:roberto.ragazzoni@oapd.inaf.it}{roberto.ragazzoni@oapd.inaf.it}
%
%
% \textbf{Professor Jean-René Roy}
%
% I worked under the direction of Jean-René Roy when he was Gemini deputy director in the early 2000s. We established and co-edited the science case for the Gemini multi-conjugate adaptive optics system GeMS. Jean-René was interim director of Gemini in 2005-2006.
%
% \href{mailto:jrroy.astro@gmail.com}{jrroy.astro@gmail.com}

\end{document}
